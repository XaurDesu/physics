%%%%%%%%%%%%%%%%%%%%%%%%%%%%%%%%%%%%%%%%%
%  A bright and image filled report style, currently set up here for use with ILM report 8600-219.
%  Contains all that is required, glossaries, content management, references and good looks.
%
% The original template (the Legrand Orange Book Template) can be found here --> http://www.latextemplates.com/template/the-legrand-orange-book
% Original author of the Legrand Orange Book Template:
% Mathias Legrand (legrand.mathias@gmail.com) 
%
% Modifications made for ILM specific reporting
% 
%
% License:
% CC BY-NC-SA 3.0 (http://creativecommons.org/licenses/by-nc-sa/3.0/)
%%%%%%%%%%%%%%%%%%%%%%%%%%%%%%%%%%%%%%%%%
 
%%%%%%%%%%%%%%%%%%%%%%%%%%%%%%%%%%%%%%%%%
% How to use this
%
% Upload a file called FrontCover.jpg to become your new front cover - into the Pictures folder
% Upload files called Heading1.jpg, Heading2.jpg etc to become your new chapter headers - into the Pictures folder
% Make sure these images are the right size to fit their locations and use good quality images
%
% Locate the variables below and set your name, title etc.
%
% If you want to change text colour on the front cover, the areas required are commented below
% If you want to modify text and border colours for your chapter headers go into the structure.tex file and replace the name of the colour (set to ) with a new colour name (find and replace ctrl+f will do this for you).
%
% Add all references into references.bib
% Cite these references by using \cite{referenceName}
%
% Commonly used acronyms or industry specific terms should be added to the glossary
% These terms may then be referenced in the text using \gls{termName}
%
% Finally put some answers in there!
%
% 
% Note: This template is set up specifically for ILM reports, it can be modified for other forms of reports
%
%%%%%%%%%%%%%%%%%%%%%%%%%%%%%%%%%%%%%%%%
 
 
%----------------------------------------------------------------------------------------
%	SET THESE VARIABLES!
%----------------------------------------------------------------------------------------

\def\mytitle{Physics - First Tome.} % Title of the ILM project
\def\ILMCode{A friendly approach.} % Unique code for the ILM project

\def\ILMCentreName{ILM centre name } % The name of the centre you're the ILM sitting at
\def\ILMCentreCode{123456/A} % Unique code the centre at which you're sitting the ILM
\def\ILMLevel{3 } % What level are you sitting with the ILM. i.e. 3, 4, 5
\def\reviewer{Reviewer's name} % You may not know this, if not use the centre's name

\def\author{Jaime Andres Torres Bermejo} % Your name.. 
\def\id{github.com/XaurDesu} % Your unique identifier

\def\date{\today } % Today's date 


 
%----------------------------------------------------------------------------------------
%	PACKAGES AND OTHER DOCUMENT CONFIGURATIONS
%----------------------------------------------------------------------------------------

\documentclass[11pt,fleqn]{book} % Default font size and left-justified equations

\usepackage[dvipsnames]{xcolor}

\input{structure} % Insert the commands.tex file which contains the majority of the structure behind the template

\makeglossaries

%--------------------------------------------------------------------------

% Glossary entries

%--------------------------------------------------------------------------
\newglossaryentry{ETN}
{
    name = {Example Term Name (ETN)},
    description = {What does this term mean? Any examples of it? Further reading? References?}
}

%--------------------------------------------------------------------------

% Document begins here

%--------------------------------------------------------------------------

\begin{document}
\renewcommand{\bibname}{References} % Adds in the link to your references


%----------------------------------------------------------------------------------------
%	TITLE PAGE
%----------------------------------------------------------------------------------------

\begingroup
\thispagestyle{empty}
\AddToShipoutPicture*{\put(0,0){\includegraphics{Pictures/FrontCover.jpg}}} % Image background
\centering
\vspace*{11.3cm}
\par\normalfont\fontsize{35}{35}\sffamily\selectfont

\begin{center}
    % List of Latex Colour names here: https://www.overleaf.com/learn/latex/Using_colours_in_LaTeX
    \textbf{\color{Apricot} \mytitle}  % Modify the name of the colour used to suit your image
    
    \textbf{\color{White}(\ILMCode)} % Modify the name of the colour used to suit your image
    
    \color{black}ILM\par % Modify the name of the colour used to suit your image
    
    \vspace*{0.5cm}
    \color{Apricot}\author % Modify the name of the colour used to suit your image
    
    (\id)\par  
\end{center}

\endgroup

%----------------------------------------------------------------------------------------
%	COPYRIGHT PAGE
%----------------------------------------------------------------------------------------


\newpage
~\vfill
\thispagestyle{empty}

\noindent \textbf{Why taking notes like this?}
\vspace{0.5cm}

\noindent Well damn, i just want to have notes that i can read easily, are non-judgemental
and let you learn at your own page, this is by no means better than a specialized textbook,
made by an actual professional with a degree in physics or some other natural science that just
so happens to involve physics. 

\vspace{1cm}

\noindent \textbf{Can i use this text in some way?}
\vspace{0.5cm}

\noindent I mean, as long as you're conscious of it's limitations and are able to work around them, i see
no reason why i would get angry about you using this text for your own means, just remember to do no harm.

\vspace{1cm}

\noindent \textbf{There's a mistake in this book! What do i do?}
\vspace{0.5cm}

\noindent Tell me what it is, just let me know and maybe even correct it yourself, i have
no reservations on making changes in case it happens to be necessary or otherwise useful.

\vspace{1cm}

\noindent \textbf{Can i share this?}
\vspace{0.5cm}

\noindent Go ahead! These notes are open source, everybody should be able to access them i think.

\vspace{1cm}

\begin{flushright}
    \textit{(Shoutout to Yenny Hernandez, who i studied physics with @Uniandes.)}
\end{flushright}


\noindent \textit{First release, \date} % Printing/edition date

%----------------------------------------------------------------------------------------
%	TABLE OF CONTENTS
%----------------------------------------------------------------------------------------

\chapterimage{Heading1.jpg} % Table of contents heading image

\pagestyle{empty} % No headers

\tableofcontents % Print the table of contents itself

%\listoftables %uncomment this if you want to print the list of tables at the start

\pagestyle{fancy} % Print headers again


%----------------------------------------------------------------------------------------
%	Glossary
%----------------------------------------------------------------------------------------

\chapterimage{Heading1.jpg} % Table of contents heading image

\printglossaries



%----------------------------------------------------------------------------------------
%	First set of related questions
%----------------------------------------------------------------------------------------

\chapterimage{Heading2.jpg}
\chapter{The process of learning physics.}

\section{So, how \textbf{AND WHY} do we learn physics, anyways?}
\begin{flushright}
    \textit{(...like, really, why?)}
\end{flushright}

Eh, it depends.

\noindent In reality, the process of learning physics is also the process of learning how to 
observe a phenomenon in a bunch of ways. It's an experimental science that is concerned with the
study of energy, materials and their mutual interactions. We can explain a bunch of things regarding
the way our universe works with it. 

In any case, learning physics is not a marathon, cramping before an exam stresses me out, and it probably does
to you too. And we should try to learn it bit by bit, going through topics slowly and surely. Rome wasn't built in
a day, and neither should you try to learn a solid understanding in one.

When learning physics, remember:
\begin{itemize}
    \item \textbf{Do exercises.} You shouldn't try to just learn by reading the notes of some other
    guy, you might not even know me. It's 
    \item \textbf{Be critical of your own solution.} you won't always have problems right your first time
    around. If something doesn't seem to make sense, it's because it is probably not right. 
    (you probably shouldn't be getting a negative velocity when trying to calculate a bike going in a
    straight line, for example)
    \item \textbf{Understand the topics enough to explain them to someone else.} If you can teach physics to
    someone else, you will probably understand it a lot better yourself, give it a try!
    \item \textbf{Be prepared.} Check your notes, do your homework, and try to study by yourself. Prepare the things that
    you need to learn before learning them becomes a point of stress.
    \item \textbf{You're not alone.} Look for help when you need it, it's not shameful.
\end{itemize}

\vspace{20px}

\section{Important considerations.}

%----------------------------------------------------------------------------
%	Second chapter
%----------------------------------------------------------------------------

\chapter{Measurements.}

\section{Uncertainty}

There is always a level of uncertainty when measuring an object, given by the instrument of
measurement we use, an atomic clock is not the same as your uncle's watch, and even though they both
measure time, there's a certain level of uncertainty caused by the device, keep it in mind when doing experimental
work.

\section{Scientific Notation}

It is usual to measure both inmense and infimal quantities of mass, time, or any other thing.
For that, we shall apply the scientific notation, both on this book and elsewhere. 

\subsubsection{EXERCISES}

\paragraph*{How many years older will you be in a thousand million seconds?}

We'll start by measuring how many seconds there are in a year.

$ \frac{1 min}{60 s} * \frac{1 h}{60min} * \frac{1 d}{24 h} * \frac{1 yr.}{365 d} = 1x10^9 s $

And now we'll take our given time on years

$ time = 1,000,000s * 1,000s = 1000x10^6 s $

\textbf{Answer:}$ 31 yrs. $


\paragraph*{How many nanoseconds does the speed of light need to travel 1ft in the void?}

$ 1 ft = 30,48 cm $

\section{Types of units}
In the realm of physics, there is a bunch of ways to categorize units, be it by what do they 
measure or how do they measure it

\section{Scalar units}



\section{Vectors}


So, given an angle $ \varphi $, we can suppouse the measurements of a two-directional vector as:
\begin{itemize}
    \item  $ A_x $
    
    $ |\vec{A}| \cos \varphi $
    \item $ A_y $
    
    $ |\vec{A}| \sin \varphi $
\end{itemize}

And given this, we can consider the vector '$\vec{A}$' as, $ \vec{A} = A_x i + A_y j $


%----------------------------------------------------------------------------
%	More sections?
%----------------------------------------------------------------------------

% Simply upload additional images Heading5.jpg, Heading6.jpg etc. into the pictures folder

\chapterimage{Heading4.jpg}
\chapter{Useful Information}

\section{Constants}
\begin{itemize}

    \item Speed of light: 
    
    $299'792.458 \frac{km}{s}$, can be approximated to $ 3x10^8 m/s $ when you're not concerned with precision.

    \item Gravity in earth: 
    
    $ 9,81 \frac{km}{s^2}$ (A bit oversimplified, but for now it should suffice)

\end{itemize}


%----------------------------------------------------------------------------------------
%	References
%----------------------------------------------------------------------------------------

\chapterimage{Heading4.jpg} % Chapter heading image

\bibliographystyle{plain} % Change this to IEEE or Harvard etc.
\bibliography{references}


\end{document}
%%%%%%%%%%%%%%%%%%%%%%%%%%%%%%%%%%%%%%%%%
%  A bright and image filled report style, currently set up here for use with ILM report 8600-219.
%  Contains all that is required, glossaries, content management, references and good looks.
%
% The original template (the Legrand Orange Book Template) can be found here --> http://www.latextemplates.com/template/the-legrand-orange-book
% Original author of the Legrand Orange Book Template:
% Mathias Legrand (legrand.mathias@gmail.com) 
%
% Modifications made for ILM specific reporting
% 
%
% License:
% CC BY-NC-SA 3.0 (http://creativecommons.org/licenses/by-nc-sa/3.0/)
%%%%%%%%%%%%%%%%%%%%%%%%%%%%%%%%%%%%%%%%%
 
%%%%%%%%%%%%%%%%%%%%%%%%%%%%%%%%%%%%%%%%%
% How to use this
%
% Upload a file called FrontCover.jpg to become your new front cover - into the Pictures folder
% Upload files called Heading1.jpg, Heading2.jpg etc to become your new chapter headers - into the Pictures folder
% Make sure these images are the right size to fit their locations and use good quality images
%
% Locate the variables below and set your name, title etc.
%
% If you want to change text colour on the front cover, the areas required are commented below
% If you want to modify text and border colours for your chapter headers go into the structure.tex file and replace the name of the colour (set to ) with a new colour name (find and replace ctrl+f will do this for you).
%
% Add all references into references.bib
% Cite these references by using \cite{referenceName}
%
% Commonly used acronyms or industry specific terms should be added to the glossary
% These terms may then be referenced in the text using \gls{termName}
%
% Finally put some answers in there!
%
% 
% Note: This template is set up specifically for ILM reports, it can be modified for other forms of reports
%
%%%%%%%%%%%%%%%%%%%%%%%%%%%%%%%%%%%%%%%%
 
 
%----------------------------------------------------------------------------------------
%	SET THESE VARIABLES!
%----------------------------------------------------------------------------------------

\def\mytitle{Physics - First Tome.} % Title of the ILM project
\def\ILMCode{A friendly approach.} % Unique code for the ILM project

\def\ILMCentreName{ILM centre name } % The name of the centre you're the ILM sitting at
\def\ILMCentreCode{123456/A} % Unique code the centre at which you're sitting the ILM
\def\ILMLevel{3 } % What level are you sitting with the ILM. i.e. 3, 4, 5
\def\reviewer{Reviewer's name} % You may not know this, if not use the centre's name

\def\author{Jaime Andres Torres Bermejo} % Your name.. 
\def\id{github.com/XaurDesu} % Your unique identifier

\def\date{\today } % Today's date 


 
%----------------------------------------------------------------------------------------
%	PACKAGES AND OTHER DOCUMENT CONFIGURATIONS
%----------------------------------------------------------------------------------------

\documentclass[11pt,fleqn]{book} % Default font size and left-justified equations

\usepackage[dvipsnames]{xcolor}

\input{structure} % Insert the commands.tex file which contains the majority of the structure behind the template

\makeglossaries

%--------------------------------------------------------------------------

% Glossary entries

%--------------------------------------------------------------------------
\newglossaryentry{ETN}
{
    name = {Example Term Name (ETN)},
    description = {What does this term mean? Any examples of it? Further reading? References?}
}

%--------------------------------------------------------------------------

% Document begins here

%--------------------------------------------------------------------------

\begin{document}
\renewcommand{\bibname}{References} % Adds in the link to your references


%----------------------------------------------------------------------------------------
%	TITLE PAGE
%----------------------------------------------------------------------------------------

\begingroup
\thispagestyle{empty}
\AddToShipoutPicture*{\put(0,0){\includegraphics{Pictures/FrontCover.jpg}}} % Image background
\centering
\vspace*{11.3cm}
\par\normalfont\fontsize{35}{35}\sffamily\selectfont

\begin{center}
    % List of Latex Colour names here: https://www.overleaf.com/learn/latex/Using_colours_in_LaTeX
    \textbf{\color{Apricot} \mytitle}  % Modify the name of the colour used to suit your image
    
    \textbf{\color{White}(\ILMCode)} % Modify the name of the colour used to suit your image
    
    \color{black}ILM\par % Modify the name of the colour used to suit your image
    
    \vspace*{0.5cm}
    \color{Apricot}\author % Modify the name of the colour used to suit your image
    
    (\id)\par  
\end{center}

\endgroup

%----------------------------------------------------------------------------------------
%	COPYRIGHT PAGE
%----------------------------------------------------------------------------------------


\newpage
~\vfill
\thispagestyle{empty}

\noindent \textbf{Why taking notes like this?}
\vspace{0.5cm}

\noindent Well damn, i just want to have notes that i can read easily, are non-judgemental
and let you learn at your own page, this is by no means better than a specialized textbook,
made by an actual professional with a degree in physics or some other natural science that just
so happens to involve physics. 

\vspace{1cm}

\noindent \textbf{Can i use this text in some way?}
\vspace{0.5cm}

\noindent I mean, as long as you're conscious of it's limitations and are able to work around them, i see
no reason why i would get angry about you using this text for your own means, just remember to do no harm.

\vspace{1cm}

\noindent \textbf{There's a mistake in this book! What do i do?}
\vspace{0.5cm}

\noindent Tell me what it is, just let me know and maybe even correct it yourself, i have
no reservations on making changes in case it happens to be necessary or otherwise useful.

\vspace{1cm}

\noindent \textbf{Can i share this?}
\vspace{0.5cm}

\noindent Go ahead! These notes are open source, everybody should be able to access them i think.

\vspace{1cm}

\begin{flushright}
    \textit{(Shoutout to Yenny Hernandez, who i studied physics with @Uniandes.)}
\end{flushright}


\noindent \textit{First release, \date} % Printing/edition date

%----------------------------------------------------------------------------------------
%	TABLE OF CONTENTS
%----------------------------------------------------------------------------------------

\chapterimage{Heading1.jpg} % Table of contents heading image

\pagestyle{empty} % No headers

\tableofcontents % Print the table of contents itself

%\listoftables %uncomment this if you want to print the list of tables at the start

\pagestyle{fancy} % Print headers again


%----------------------------------------------------------------------------------------
%	Glossary
%----------------------------------------------------------------------------------------

\chapterimage{Heading1.jpg} % Table of contents heading image

\printglossaries



%----------------------------------------------------------------------------------------
%	First set of related questions
%----------------------------------------------------------------------------------------

\chapterimage{Heading2.jpg}
\chapter{The process of learning physics.}

\section{So, how \textbf{AND WHY} do we learn physics, anyways?}
\begin{flushright}
    \textit{(...like, really, why?)}
\end{flushright}

Eh, it depends.

\noindent In reality, the process of learning physics is also the process of learning how to 
observe a phenomenon in a bunch of ways. It's an experimental science that is concerned with the
study of energy, materials and their mutual interactions. We can explain a bunch of things regarding
the way our universe works with it. 

In any case, learning physics is not a marathon, cramping before an exam stresses me out, and it probably does
to you too. And we should try to learn it bit by bit, going through topics slowly and surely. Rome wasn't built in
a day, and neither should you try to learn a solid understanding in one.

When learning physics, remember:
\begin{itemize}
    \item \textbf{Do exercises.} You shouldn't try to just learn by reading the notes of some other
    guy, you might not even know me. It's 
    \item \textbf{Be critical of your own solution.} you won't always have problems right your first time
    around. If something doesn't seem to make sense, it's because it is probably not right. 
    (you probably shouldn't be getting a negative velocity when trying to calculate a bike going in a
    straight line, for example)
    \item \textbf{Understand the topics enough to explain them to someone else.} If you can teach physics to
    someone else, you will probably understand it a lot better yourself, give it a try!
    \item \textbf{Be prepared.} Check your notes, do your homework, and try to study by yourself. Prepare the things that
    you need to learn before learning them becomes a point of stress.
    \item \textbf{You're not alone.} Look for help when you need it, it's not shameful.
\end{itemize}

\vspace{20px}

\section{Important considerations.}

%----------------------------------------------------------------------------
%	Second chapter
%----------------------------------------------------------------------------

\chapter{Measurements.}



\section{Uncertainty}

There is always a level of uncertainty when measuring an object, given by the instrument of
measurement we use, an atomic clock is not the same as your uncle's watch, and even though they both
measure time, there's a certain level of uncertainty caused by the device, keep it in mind when doing experimental
work. We can express this with the symbol ' $\pm$ '

for example, if we had a milimeter of uncertainty on a measurement of 9,5 cm,
we could indicate it as such:
\begin{equation}
    l = 9,5 \pm 0,1 cm
\end{equation}

\section{Scientific Notation}

It is usual to measure both inmense and infimal quantities of mass, time, or any other thing.
For that, we shall apply the scientific notation, both on this book and elsewhere. 

\subsubsection{EXERCISES}

\paragraph*{How many years older will you be in a thousand million seconds?}

We'll start by measuring how many seconds there are in a year.

$ \frac{1 min}{60 s} * \frac{1 h}{60min} * \frac{1 d}{24 h} * \frac{1 yr.}{365 d} = 1x10^9 s $

And now we'll take our given time on years

$ time = 1,000,000s * 1,000s = 1000x10^6 s $

\textbf{Answer:}$ 31 yrs. $


\paragraph*{How many nanoseconds does the speed of light need to travel 1ft in the void?}

$ 1 ft = 30,48 cm $

Even though not specifically notified in the exercise, we must remember that:

\begin{gather}
    1m = 100cm    
\end{gather}

\section{Types of units}
In the realm of physics, there is a bunch of ways to categorize units, be it by what do they 
measure or how do they measure it

\section{Scalar units}

A scalar unit is defined as an unit with magnitude, but no direction. Such measurements
aren't really concerned with being tied with a specific force acting over a particle. 

Examples of such units include:
\begin{itemize}
    \item Natural numbers.
    \item Constants.
    \item Acceleration
\end{itemize}

\section{Vectors}

A vector is a measurement with both a magnitude and a direction. that can exist on
a specific set of dimensions. On this course, we won't be concerning ourselves with hyperplanes
and might sparingly cover 3-dimensional planes, they would both be useful for you to learn eventually, so i encourage
you to learn about them on your own accord, however. Our examples will be two-dimensional, for a simple, 
introductory example, consider the following vector; $ \vec{A} = 
\begin{pmatrix}
    3 \\
    2
\end{pmatrix} $ :

\begin{center}
 \includegraphics*[scale=0.5]{vec_1.png}

 \textit{Vector with x coordinates '3' and y coordinates '2'}
\end{center}


This is what we call a cartesian vector, for it's magnitudes are defined by the coordinates in a
cartesian plane, however, we can turn it into a polar vector, measured by it's magnitude and 
angle, through the following formulas:

\begin{gather}
    |\vec{A}| = \sqrt{A_x^2+A_y^2} \\
    \tan \theta = \frac{A_y}{A_x}
\end{gather}

\paragraph{Exercise: Turning our example vector 'A' into a Polar vector.}
\textit{Magnitude}
\begin{gather}
    |\vec{A}| = \sqrt{3^2+2^2}
    |\vec{A}| = \sqrt{9+4} = \sqrt{12} = 2\sqrt{3} 
\end{gather}
\textit{Angle}
\begin{gather}
    \tan \theta = \frac{2}{3} \\
    \theta = \tan^-1(\frac{2}{3}) \\
    \theta = 33,69^o 
\end{gather}

\noindent Equally, we can turn a polar vector into a cartesian one. 

So, given an angle $ \theta $ and a magnitude $ |\vec{A}| $, we can suppouse the measurements of a two-directional vector as:
\begin{gather}
    A_x  = |\vec{A}| \cos \theta \\
    A_y = |\vec{A}| \sin \theta
\end{gather}

And given this, we can consider the vector '$\vec{A}$' as, $ \vec{A} = A_x i + A_y j $, where 'i' and
'j', are what we're going to call a \textbf{unit vector}, a vector in a specific direction that has a value of 1.
(this is considered an identity value for multiplication operations, and lets us do some vector sum 
operations more intuitively)


\paragraph{Exercise: Turning our example vector 'A' back into a cartesian vector.}
\begin{gather}
    A_x  = |2\sqrt{3}| \cos 33,69^o = 2,88 \approxeq 3 \\
    A_y = |2\sqrt{3}| \sin 33,69^o = 1,92  \approxeq 2
\end{gather}

As it might be evident, the conversion isn't exact, this happens because of the 
angle not being an exact conversion, this will happen when you disregard part of a value
for whatever reason. The more exact of a measurement you keep, the less uncertainty you will
end up with.

\subsection{Vector sum}


We can take any $ \vec{a} $ and $ \vec{b} $ vectors on the same space and add them to each other
in the form:

\begin{equation}
    \begin{bmatrix}
        a_1 \\
        a_2 \\
        a_3 \\
    \end{bmatrix}
    +
    \begin{bmatrix}
        b_1 \\
        b_2 \\
        b_3 \\
    \end{bmatrix}
    = 
    \begin{pmatrix}
        a_1 + b_1\\
        a_2 + b_2\\
        a_3 + b_3\\
    \end{pmatrix}
\end{equation}

Such form remains in the case we can do subtraction, which is expressed on the equation:

\begin{equation}
    \begin{bmatrix}
        a_1 \\
        a_2 \\
        a_3 \\
    \end{bmatrix}
    -
    \begin{bmatrix}
        b_1 \\
        b_2 \\
        b_3 \\
    \end{bmatrix}
    = 
    \begin{pmatrix}
        a_1 - b_1\\
        a_2 - b_2\\
        a_3 - b_3\\
    \end{pmatrix}
\end{equation}

This kind of operations have certain properties, shown as:
\begin{gather}
    (\alpha + \beta)\vec[v] = \alpha\vec{v} + \beta\vec{v} \\
    \vec{v} * 1 = \vec{v} \\
    \vec{v} * \vec{0} = \vec{0} \\
    \beta \vec{v} = 
    \begin{pmatrix}
        \beta a_1\\
        \beta a_2\\
        \beta a_3\\
    \end{pmatrix}
\end{gather}

Two vectors $ \vec{a} $ and $ \vec{b} $ are equal if and only if:
\begin{equation}
    \begin{cases}
        \vec{a} \exists \mathbb{R}^3 \\
        \vec{b} \exists \mathbb{R}^3
    \end{cases}
    \implies
    \begin{pmatrix}
        a_1 = b_1\\
        a_2 = b_2\\
        a_3 = b_3\\
    \end{pmatrix}
    \text{
        \textit{note: this can be generalized to 'n' dimensions larger than 0}}
\end{equation}

in either case,  $ \vec{0} $ is the identity of the operation, therefore:
\begin{equation}
    \vec{a} + \vec{0} = \vec{a}
\end{equation}


\subsection{Scalar/dot product}
We can multiply vectors between each other with the following formula:

$$ |\vec{A}| \centerdot |\vec{B}| = |\vec{A}||\vec{B}|\cos \theta $$

We can also write it as such for cartesian vectors:
$$ A_x B_x + A_y B_y + A_z B_z $$

This is a commutative operation, that won't be affected by neither A nor B's 


\subsection{Cross/Vectorial product}

We can define the formula for the cross product of two vectors in a 3-dimensional space
as:
\begin{equation}
    \vec{A} x \vec{B} = 
    \begin{pmatrix}
        (A_y - B_z - A_z B_y) i \\
        (A_x B_z - B_x A_z) j \\
        (A_x B_y - B_x A_y) k   
    \end{pmatrix}
\end{equation}

This will come handy when studying Torque;

\subsection{Solved Exercises}
\subsubsection{Vector Sum}
\paragraph{An espeologist explores a cave and follows a }

\subsection{Dot/Cross Product}


\subsection{Scientific Notation}
\paragraph{how many nanoseconds does a ray of light require to travel 3 meters in the void?}

\textit{Solution}


\paragraph{Estimate how many gallons of gasoline are consumed by private cars in Colombia in a year.
Assume that on average there is one car for every 10 inhabitants and that on average each car travels $10^4$ km per year, with an average engine efficiency of 25 km/gallon.}

\indent \textit{Solution.}
\begin{gather}
    \text{Population of Colombia as of 2020}= 51520000 \\
    \text{Number of cars}= 51520000/10 = 5152000 \text{ } cars\\
    \text{Kilometers traveled}=5152000 \text{ } cars * 10^4 \frac{km}{yr} = 51520000000\frac{km}{yr} = 5,152x10^{10}\frac{km}{yr}\\
    \text{Gallons per year}= \frac{5,152x10^{10}\frac{km}{yr}}{25\frac{km}{gallon}}=2060800000\frac{\text{gallon}}{yr} = 2,0608x10^{9}\frac{\text{gallon}}{yr}
\end{gather}

%----------------------------------------------------------------------------
%	More sections?
%----------------------------------------------------------------------------

% Simply upload additional images Heading5.jpg, Heading6.jpg etc. into the pictures folder

\chapterimage{Heading4.jpg}
\chapter{Basic Kinematics}

\section{One-Dimensional Movement}
One-dimensional movement is probably the most basic kind of movement, as unbound by the expectations of 
more than two directions to move to, it allows itself to simply define the route from a point 'A' to a 
point 'B' in a straight line. 



Such movement can be defined as:
\begin{equation}
    \Delta x = x_2 - x_1   
\end{equation}

medium velocity on this kind of movement can be calculated as:

\begin{equation}
    v_{med} = \frac{\Delta x}{\Delta y} = \frac{x_2 - x_1}{t_2 - t_1}
\end{equation}

Keep in mind, we can define negative velocity in such systems to move in a
negative direction (i.e moving from B=7 to A=3, for example.) and besides that
\textbf{This is not the same as speed, because speed is concerned with how fast the object moves
while this is concerned with how fast the object arrives somewhere, and that's not the same.}

\section{instantaneous speed and velocity}

It should now be evident that we can assume a straight line that can be interpreted as the velocity of an
object, as we could see in this two-dimensional graph.

\begin{center}
    \includegraphics[scale = 0.6]{pendant.png}    
\end{center}

We define the instantaneous velocity of an object as a derivative of the form:
\begin{equation}
    v = \lim_{\Delta t \to 0} \frac{\Delta x}{\Delta t} = \frac{dx}{dt}   
\end{equation}

The simplest formula for such a calculation is, however:

\begin{equation}
    x = x_0 + vt
\end{equation}


\section{Acceleration}

Of course, not all objects will move at a constant pace, most times there will be a 
change in speed happening at some moment in their movement. This is a scalar 
variable that can be used to calculate the overall distance when taking into account
such changes on movement. We can calculate the acceleration of an object as follows:

\begin{equation}
    a = \frac{\Delta v}{\Delta t}
\end{equation}

we can integrate distance as a function of acceleration as it follows, as well:

\begin{equation}
    \int_{t}^{0} v \,dt = \int_{x}^{x_0} \,dx \\ \text{\textit{ Note: we can integrate the result of this equation to prove correctness.}}
\end{equation}

Through a mathematical proof that 

\subsection{Constant Acceleration}

An object can be said to have constant acceleration when acceleration is not a variable, but a 
constant. When working on such a system we can affirm velocity as:

\begin{equation}
    at = v - v_0
\end{equation}

and we can indicate distance as:

\begin{equation}
    x = x_0 + v_0 t + \frac{1}{2} at^2
\end{equation}

From these two we can gather the following formulas:

\begin{itemize}
    \item $v = v_0 + at$
    \item $t = \frac{v-v_0}{a}$
    \item $ v^2 = v^2_0 + 2a(x-x_0) \text{\textit{If we don't know time}} $
    \item $ x - x_0 = (\frac{v+v_0}{2})t \text{\textit{If we don't know acceleration}} $
\end{itemize}

\section{Exercises}
\subsection*{One-dimensional movement}


\subsection{Acceleration}
\paragraph{what's the maximum possible height of a marker being thrown at a
velocity of $ 0,5 \frac{m}{s}$ ?}

We know that:
\begin{itemize}
    \item gravity (acceleration): $ -9,8 \frac{m}{s^2} $
    \item initial position: 0
    \item initial velocity = $ 0,5 \frac{m}{s} $
\end{itemize}

\begin{gather}
    v^2 = v^2_0 + 2a(x-x_0) \\
    0 = v_0^2 - 2g(y-0) \\
    v_0^2 = 2gy \\
    y = \frac{v_0^2}{2g} \\ 
    y = - \frac{0,5\frac{m}{s}^2}{2(9,8\frac{m}{s^2})}
\end{gather}


\chapterimage{Heading4.jpg}
\chapter{Useful Information}

\section{Constants}
\begin{itemize}

    \item Speed of light: 
    
    $299'792.458 \frac{km}{s}$, can be approximated to $ 3x10^8 m/s $ when you're not concerned with precision.

    \item Gravity in earth: 
    
    $ 9,81 \frac{km}{s^2}$ (A bit oversimplified, but for now it should suffice)

\end{itemize}

\section{Definitions}
\begin{itemize}
    \item Delta: 
    
    noted as $ \Delta $, it indicates change in a variable, taking $ x $ as the variable
    that changes, $ x_0 $ as it initial state and $ x_f $ as the final one, we can indicate how much it
    changes as follows: $$ \Delta x = x_f - x_0 $$
\end{itemize}
%----------------------------------------------------------------------------------------
%	References
%----------------------------------------------------------------------------------------

\chapterimage{Heading4.jpg} % Chapter heading image

\bibliographystyle{plain} % Change this to IEEE or Harvard etc.
\bibliography{references}


\end{document}